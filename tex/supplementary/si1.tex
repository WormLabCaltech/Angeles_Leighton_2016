\documentclass{article}
\usepackage[utf8]{inputenc}
\usepackage{hyperref}
\usepackage[T1]{fontenc}
\usepackage{booktabs}

\newcommand{\ra}[1]{\renewcommand{\arraystretch}{#1}}
\newcommand{\cel}{\emph{C.~elegans}}

\title{Kallisto Commands, Brief Description of All Other SI}
\author{David Angeles-Albores\textsuperscript{1,$\dagger{}$}
\and{}
Daniel H.W. Leighton\textsuperscript{2,$\dagger{}$}
\and{}
Tiffany Tsou\textsuperscript{1}
\and{}
Tiffany H. Khaw\textsuperscript{1}
\and{}
Igor Antoshechkin\textsuperscript{3}
\and{}
Paul W. Sternberg\textsuperscript{1,*}
}



\begin{document}

 \maketitle
 \tableofcontents

 \section{Kallisto Commands}

 The Kallisto Commands were written in an automated fashion using a Python script. The shell script can be found at the following url: \url{https://github.com/WormLabCaltech/Angeles_Leighton_2016/blob/master/input/kallisto_commands.sh}.

 The resulting TPM files were analysed using Sleuth via a customized R script, \texttt{aging\_rna\_seq\_analysis.R}. This script can be found here: \url{https://github.com/WormLabCaltech/Angeles_Leighton_2016/blob/master/rdocs/aging_rna_seq_analysis.R}.

 The results from Sleuth were analyzed using Python in a Jupyter notebook. All Python code can be found in the \texttt{src} directory of our Github repository.


\section{Supplementary Files 2, 3, 4}

Supplementary Files 2, 3, and 4 contain the output of Sleuth after differential expression analysis. The files contain the columns detailed in Table~\ref{tab:sleuth_columns}. Supplementary File 2 contains what can be loosely interpreted as the fold-change of each gene between young and old adult worms. Supplementary File 3 contains what can be loosely interpreted as the fold-change of each gene between wild-type and \emph{fog-2} worms. Supplementary File 4 contains the genes that changed differently as wild-type worms aged and \emph{fog-2} worms aged. For Supplementary File 4, the `b' column should be interpreted as the change in the aging log-fold change. That is, an aging \emph{fog-2} animal would have a log-fold change in a given gene \emph{X} equal to $b_\mathrm{age} + b_\mathrm{interaction}$, whereas an aging wild-type animal would have a log-fold change equal to $b_\mathrm{age}$.

\begin{table*}
  \centering{}
  \ra{1.3}
   \begin{tabular}{@{}ll@{}}
   \toprule{}\\
   Column Name & Description\\
  \toprule{}\\
   target\_id & Gene identifier, usually unique to each isoform in \cel{}\\
   pval & P-value indicating significance of differential expression.\\
    & NOT FDR Corrected. NOT used to identify D.E. genes\\
   qval & P-value after correcting for multiple hypothesis testing.\\
    & This column is used to establish statistical significance\\
   b & The slope of the regression along a specified axis.\\
    & \textbf{Loosely} interpreted as natural log of the fold-change.\\
    se\_b & Standard error of b.\\
    mean\_obs & Mean of the logarithm of the observed counts.\\
    var\_obs & Variance of the logarithm of the observed counts.\\
    tech\_var & Technical variance of the logarithm of the observed counts.\\
    sigma\_sq & Biological variance of the logarithm of the observed counts.\\
    smooth\_sigma\_sq & Smoothed biological variance of the logarithm of the observed counts.\\
    ens\_gene & WormBase Identifier (not isoform specific).\\
    ext\_gene & Human-readable Gene Name\\
   \bottomrule{}
 \end{tabular}
\caption{List of columns in file SI1--3.}
\label{tab:sleuth_columns}
\end{table*}


\end{document}
